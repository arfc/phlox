\documentclass[11pt,letterpaper]{article}

\addtolength{\oddsidemargin}{-.875in}
\addtolength{\evensidemargin}{-.875in}
\addtolength{\textwidth}{1.75in}

\addtolength{\topmargin}{-.875in}
\addtolength{\textheight}{1.75in}

\usepackage[utf8]{inputenc}
\usepackage{caption} % for table captions
\usepackage{amsmath} % for multi-line equations and piecewises
\DeclareMathOperator{\sign}{sign}
\usepackage{graphicx}
\usepackage{relsize}
\usepackage{xspace}
\usepackage{verbatim} % for block comments
\usepackage{subcaption} % for subfigures
\usepackage{enumitem} % for a) b) c) lists
\newcommand{\Cyclus}{\textsc{Cyclus}\xspace}%
\newcommand{\Cycamore}{\textsc{Cycamore}\xspace}%
\newcommand{\deploy}{\texttt{d3ploy}\xspace}%
\newcommand{\Deploy}{\texttt{D3ploy}\xspace}%
\usepackage{tabularx}
\usepackage{color}
\usepackage{multirow}
\usepackage{float} 
\usepackage[acronym,toc]{glossaries}
%\include{acros}
\definecolor{bg}{rgb}{0.95,0.95,0.95}
\newcolumntype{b}{X}
\newcolumntype{f}{>{\hsize=.15\hsize}X}
\newcolumntype{s}{>{\hsize=.5\hsize}X}
\newcolumntype{m}{>{\hsize=.75\hsize}X}
\newcolumntype{r}{>{\hsize=1.1\hsize}X}
\usepackage{titling}
\usepackage[hang,flushmargin]{footmisc}
\renewcommand*\footnoterule{}
\usepackage{tikz}

\usetikzlibrary{shapes.geometric,arrows}
\tikzstyle{process} = [rectangle, rounded corners, 
minimum width=1cm, minimum height=1cm,text centered, draw=black, 
fill=blue!30]
\tikzstyle{arrow} = [thick,->,>=stealth]

\graphicspath{}
\title{Micro Modular Reactor - USNC}
\author{Roberto E. Fairhurst Agosta}

\begin{document}
%	\begin{titlepage}
%		\maketitle
%		\thispagestyle{empty}
%	\end{titlepage}

\section{Description}	

\section{Serpent2 Model}

\subsection{FCM}
\label{sub:FCM}

\begin{figure}[H]
	\centering
	\begin{subfigure}[t]{0.4\textwidth}
		\centering
		\includegraphics[width=\linewidth]{figures/FCM_geom1.png} 
		\caption{XY cross section.}
		\label{fig:FCM_xy}
	\end{subfigure}
	\vspace{1cm}
	\begin{subfigure}[t]{0.4\textwidth}
		\centering
		\includegraphics[width=\linewidth]{figures/FCM_geom2.png} 
		\caption{YZ cross section.}
		\label{fig:FCM_yz}
	\end{subfigure}
	\hfill
	\caption{FCM Serpent2 model geometry for different cross-sections.}
	\label{fig:FCM}
\end{figure}

\subsection{Fuel Assembly}
\label{sub:FuelAssembly}

\begin{figure}[H]
	\centering
	\includegraphics[width=0.4\linewidth]{figures/fuel_block_geom1.png} 
	\hfill
	\caption{Fuel Assembly Serpent2 model geometry.}
	\label{fig:FuelAssembly}
\end{figure}

\subsection{Control Rod Assembly}
\label{sub:ControlRodAssembly}

\begin{figure}[H]
	\centering
	\includegraphics[width=0.4\linewidth]{figures/control_block_geom1.png} 
	\hfill
	\caption{Control Rod Assembly Serpent2 model geometry.}
	\label{fig:ControlRodAssembly}
\end{figure}

\subsection{Central Assembly}
\label{sub:CentralAssembly}

\begin{figure}[H]
	\centering
	\includegraphics[width=0.4\linewidth]{figures/central_block_geom1.png}
	\hfill
	\caption{Central Assembly Serpent2 model geometry.}
	\label{fig:CentralAssembly}
\end{figure}

\subsection{MMR Full Core}
\label{sub:Full}

\begin{figure}[H]
	\centering
	\includegraphics[width=0.45\linewidth]{figures/MMR_full_stack_geom1.png}
	\hfill
	\caption{XY cross section of the full core Serpent2 model geometry.}
	\label{fig:FullXY}
\end{figure}

\begin{figure}[H]
	\centering
	\includegraphics[width=0.6\linewidth]{figures/MMR_full_stack_geom2.png}
	\hfill
	\caption{YZ cross section of the full core Serpent2 model geometry.}
	\label{fig:FullYZ}
\end{figure}

\pagebreak
\bibliographystyle{plain}
\bibliography{bibliography}

\end{document}
