\documentclass[11pt,letterpaper]{article}

\addtolength{\oddsidemargin}{-.875in}
\addtolength{\evensidemargin}{-.875in}
\addtolength{\textwidth}{1.75in}

\addtolength{\topmargin}{-.875in}
\addtolength{\textheight}{1.75in}

\usepackage[utf8]{inputenc}
\usepackage{caption} % for table captions
\usepackage{amsmath} % for multi-line equations and piecewises
\DeclareMathOperator{\sign}{sign}
\usepackage{graphicx}
\usepackage{relsize}
\usepackage{xspace}
\usepackage{verbatim} % for block comments
\usepackage{subcaption} % for subfigures
\usepackage{enumitem} % for a) b) c) lists
\newcommand{\Cyclus}{\textsc{Cyclus}\xspace}%
\newcommand{\Cycamore}{\textsc{Cycamore}\xspace}%
\newcommand{\deploy}{\texttt{d3ploy}\xspace}%
\newcommand{\Deploy}{\texttt{D3ploy}\xspace}%
\usepackage{tabularx}
\usepackage{color}
\usepackage{multirow}
\usepackage{float} 
\usepackage[acronym,toc]{glossaries}
%\include{acros}
\definecolor{bg}{rgb}{0.95,0.95,0.95}
\newcolumntype{b}{X}
\newcolumntype{f}{>{\hsize=.15\hsize}X}
\newcolumntype{s}{>{\hsize=.5\hsize}X}
\newcolumntype{m}{>{\hsize=.75\hsize}X}
\newcolumntype{r}{>{\hsize=1.1\hsize}X}
\usepackage{titling}
\usepackage[hang,flushmargin]{footmisc}
\renewcommand*\footnoterule{}
\usepackage{tikz}

\usetikzlibrary{shapes.geometric,arrows}
\tikzstyle{process} = [rectangle, rounded corners, 
minimum width=1cm, minimum height=1cm,text centered, draw=black, 
fill=blue!30]
\tikzstyle{arrow} = [thick,->,>=stealth]

\graphicspath{}
\title{Micro Modular Reactor (MMR) - USNC}
\author{Roberto E. Fairhurst Agosta}

\begin{document}
%	\begin{titlepage}
%		\maketitle
%		\thispagestyle{empty}
%	\end{titlepage}

\section{Description}

The Micro Modular Reactor (MMR) concept integrates power production, power conversion, and electricity generation in a single unit, making it suitable for use in remote areas. 
The MMR uses fully ceramic micro-encapsulated (FCM) fuel, which is a form of TRISO fuel, and He (or CO$_2$) as coolant.
The power of this reactor ranges between 10 to 40 MW-thermal.
The MMR design results attractive for replacing diesel generators in remote areas.
As it could be factory produced, it could generate power at economically competitive prices \cite{hawari_development_2018}.

\section{Serpent2 Model}

\subsection{TRISO particles and fuel compact}

The input file that models the fuel compact is \textit{FCM}, shown in Figure \ref{fig:FCM}.
Table \ref{tab:compact} summarizes the characteristics of the TRISO particles and the fuel compact modeled in Serpent.

Further assumptions:
\begin{itemize}
	\item helium gap: 0.01 cm.
	\item compact length: 4 cm.
	\item helium temperature: 700 K.
	\item helium pressure: 3 MPa \cite{hawari_development_2018}.
\end{itemize}

	\begin{figure}[htbp!]
		\centering
		\begin{subfigure}[t]{0.4\textwidth}
			\centering
			\includegraphics[width=\linewidth]{figures/FCM_geom1.png} 
			\caption{XY-plane view.}
			\label{fig:FCM_xy}
		\end{subfigure}
		\begin{subfigure}[t]{0.4\textwidth}
			\centering
			\includegraphics[width=\linewidth]{figures/FCM_geom2.png}
			\caption{YZ-plane view.}
			\label{fig:FCM_yz}
		\end{subfigure}
		\hfill
		\caption{FCM Serpent model geometry for different planes.}
		\label{fig:FCM}
	\end{figure}

	\begin{table}[htbp!]
		\centering
	    \caption{TRISO and Fuel Compact Characteristics.}
	    \label{tab:compact}
		\begin{tabular}{l|l|l}
		\hline
		Characteristic              & Value             & Reference/Assumption \\ \hline
		Fuel                        & UO$_2$ + UC$_{0.5}$O$_{1.5}$ (50/50\%)  & Section 1 of \cite{hawari_development_2018} \\
		Enrichment (average)        & 12.0 wt\%         & Section 1 of \cite{hawari_development_2018}  \\
		Kernel radius               & 400 $\mu$m        & Table 5 of \cite{hawari_development_2018}  \\
		Buffer thickness            & 75 $\mu$m         & Table 5 of \cite{hawari_development_2018}  \\
		IPyC thickness              & 35 $\mu$m         & Table 5 of \cite{hawari_development_2018}  \\
		SiC thickness               & 35 $\mu$m         & 36.7 $\mu$m in table 5 of \cite{hawari_development_2018}  \\
		OPyC thickness              & 20 $\mu$m         & Table 5 of \cite{hawari_development_2018}  \\
    	Kernel density              & 10.8 g/cm$^3$     & Table 5 of \cite{hawari_development_2018}  \\
		Buffer density              & 0.98 g/cm$^3$     & Table 5 of \cite{hawari_development_2018}  \\
		IPyC density                & 1.85 g/cm$^3$     & Table 5 of \cite{hawari_development_2018}  \\
		SiC density                 & 3.2  g/cm$^3$     & Table 5 of \cite{hawari_development_2018}  \\
		OPyC density                & 1.85 g/cm$^3$     & 1.86 g/cm$^3$ in table 5 of \cite{hawari_development_2018}  \\
		SiC Matrix density          & 3.2 g/cm$^3$      & Table 1 of \cite{hawari_development_2018}  \\
		Packing Fraction (average)  & 39.9 \%           & 40\% in Table 1 of \cite{hawari_development_2018}  \\
		Compact diameter            & 2.14 cm           & Fuel pin diameter is 2.17 cm in Section 1 of \cite{hawari_development_2018}  \\
		Helium gap                  & 0.01 cm           & The gap is an assumption. \\
		Compact length              & 4.0 cm            & Value chosen arbitrily to model the compact. \\ 
        Helium density           	& 2.0526 kg/m$^3$   & --> HERE <--  \\
        Block graphite density      & 1.75 g/cm$^3$     &   \\ \hline

		\end{tabular}
	\end{table}

\subsection{Assemblies and full core}

Three different types assemblies make up the core, the fuel assembly, the control rod assembly, and a central assembly.

The input file that models the fuel assembly and the control rod assembly are \textit{fuel\_block} and \textit{control\_block}, respectively, Figure \ref{fig:assemblies}.

The input file that describes the full core is \textit{fullcore}, Figure \ref{fig:full}.

Table \ref{tab:fuel} and \ref{tab:control} summarize the characteristics of the fuel and the control rod assemblies.

The central assembly has one 12 cm-diameter control rod.

Most of the values adopted in the model come from \cite{hawari_development_2018}.
Knowing the fuel assembly flat-to-flat distance, and the number of fuel and cooling channels we estimate the flat-to-flat distance of the hexagonal lattice to be 6.4 cm.

The fuel column is 63.24 cm long, and has a graphite bottom and top reflector of 2.38 cm.
I assume that the fuel column has the length of the fuel assembly 68 cm.
The model does not include the reflectors in the fuel column.
Section 1 of \cite{hawari_development_2018} does not mention the height of the core. 
\cite{venneri_neutronic_2015} considers a reactor height of 2 meters.
Our model considers 3 fuel assemblies stacked on top of each other, reaching a core height of 204 cm.

\cite{hawari_development_2018} does not specify the height of the reflector.
\cite{venneri_neutronic_2015} uses a top and bottom reflector of 50 cm. 
Our model assumes a top and bottom reflector of 50 cm.

The core counts with a radial reflector of graphite and another one of BeO.
The radial reflector of graphite has a diameter of 248.62 cm and the reflector of BeO has a diameter of 268.62 cm\cite{hawari_development_2018}.
The model considers diameters of 248 cm and 268 cm, respectively.

See Section \ref{sec:comments} for further discussion on the assumptions.

	\begin{figure}[htbp!]
		\centering
		\begin{subfigure}[t]{0.4\textwidth}
			\centering
			\includegraphics[width=\linewidth]{figures/fuel_block_geom1.png} 
			\caption{Fuel assembly serpent model.}
		\end{subfigure}
		\begin{subfigure}[t]{0.4\textwidth}
			\centering
			\includegraphics[width=\linewidth]{figures/control_block_geom1.png}
			\caption{Control rod assembly serpent model.}
		\end{subfigure}
		\hfill
		\caption{Different assemblies geometry.}
		\label{fig:assemblies}
	\end{figure}

	\begin{figure}[htbp!]
		\centering
		\begin{subfigure}[t]{0.4\textwidth}
			\centering
			\includegraphics[width=\linewidth]{figures/fullcore_geom1.png} 
			\caption{XY-plane view.}
		\end{subfigure}
		\begin{subfigure}[t]{0.4\textwidth}
			\centering
			\includegraphics[width=\linewidth]{figures/fullcore_geom2.png}
			\caption{YZ-plane view.}
		\end{subfigure}
		\hfill
		\caption{Full core Serpent model for different planes.}
		\label{fig:full}
	\end{figure}

	\begin{table}[htbp!]
		\centering
	    \caption{Fuel Assembly Characteristics.}
	    \label{tab:fuel}
		\begin{tabular}{l|l}
		\hline
		Characteristic                   & Value         \\ \hline
		Block pitch (flat-to-flat)       & 30 cm         \\
		Block graphite density           & 1.75 g/cm$^3$ \\
		Number of fuel holes             & 54            \\
		Fuel hole diameter               & 2.16 cm       \\
		Number of coolant holes          & 19            \\
		Coolant hole radius        		 & 1.54 cm       \\
		Flat-to-flat hexagonal lattice   & 6.4 cm        \\
		Fuel length                      & 68 cm         \\ \hline
		\end{tabular}
	\end{table}

	\begin{table}[htbp!]
		\centering
	    \caption{Control Rod Assembly Characteristics.}
	    \label{tab:control}
		\begin{tabular}{l|l}
		\hline
		Characteristic                   & Value         \\ \hline
		Block pitch (flat-to-flat)       & 30 cm         \\
		Block graphite density           & 1.75 g/cm$^3$ \\
		Number of fuel holes             & 48            \\
		Fuel hole diameter               & 2.16 cm       \\
		Number of coolant holes          & 18            \\
		Large coolant hole radius        & 1.54 cm       \\
		Flat-to-flat hexagonal lattice   & 6.4 cm        \\
		Fuel length                      & 68 cm         \\ 
		Control rod diameter             & 8 cm          \\ \hline
		\end{tabular}
	\end{table}

\section{Results}

Table \ref{tab:results1} shows the results of the eigenvalue calculations in Serpent of the FCM and the full core.

Figure \ref{fig:results2} presents the variation of $k_{eff}$ and the mass of $U_{235}$ with the burn up.

For the depletion calculations, we consider a reactor power of 10 MW-th.

	\begin{table}[htbp!]
		\centering
	    \caption{$k_{eff}$ for different configurations.}
	    \label{tab:results1}
		\begin{tabular}{l|l}
		\hline
		             & $k_{eff}$ (analog)  \\ \hline
		FCM          & 1.53402 +/- 0.00084 \\ 
		fullcore     & 1.13582 +/- 0.00052 \\ \hline

		\end{tabular}
	\end{table}

%Depeltions calculations
	\begin{figure}[htbp!]
		\centering
		\begin{subfigure}[t]{0.4\textwidth}
			\centering
			\includegraphics[width=\linewidth]{figures/Keff.png}
			\caption{$k_{eff}$ of the full core for different burn ups.}
		\end{subfigure}
		\begin{subfigure}[t]{0.4\textwidth}
			\centering
			\includegraphics[width=\linewidth]{figures/mU235.png}
			\caption{$U_{235}$ mass of the full core for different burn ups.}
		\end{subfigure}
		\hfill
		\caption{Burnup calculation results.}
		\label{fig:results2}
	\end{figure}

\section{Comments}
\label{sec:comments}

This section discusses a few discrepancies between the adopted model and models from other documents.

The values for the TRISO particles are not equal to the values found in other documents such as \cite{venneri_neutronic_2015}, but are very close.

While \cite{venneri_neutronic_2015} uses a fuel compact of 0.7 cm of radius, \cite{hawari_development_2018} uses a fuel pin diameter of 2.17 cm. The model adopts a value of 2.16 cm.
\cite{venneri_neutronic_2015} uses a coolant hole of 0.4 cm-radius, and a 17 mm center to center distance hexagonal lattice.
Section 3 of the same document defines the core with graphite pancake-like blocks.
The blocks are 25 cm thick quarter disks with a diameter of 200 cm and a total length of 200 cm, surrounded by a 50 cm thick beryllium oxide reflectors.
The overall core assembly size is 300 cm in diameter and 300 cm in length.

\pagebreak
\bibliographystyle{plain}
\bibliography{bibliography}

\end{document}
