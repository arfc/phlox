\documentclass[11pt,letterpaper]{article}

\addtolength{\oddsidemargin}{-.875in}
\addtolength{\evensidemargin}{-.875in}
\addtolength{\textwidth}{1.75in}

\addtolength{\topmargin}{-.875in}
\addtolength{\textheight}{1.75in}

\usepackage[utf8]{inputenc}
\usepackage{caption} % for table captions
\usepackage{amsmath} % for multi-line equations and piecewises
\DeclareMathOperator{\sign}{sign}
\usepackage{graphicx}
\usepackage{relsize}
\usepackage{xspace}
\usepackage{verbatim} % for block comments
\usepackage{subcaption} % for subfigures
\usepackage{enumitem} % for a) b) c) lists
\newcommand{\Cyclus}{\textsc{Cyclus}\xspace}%
\newcommand{\Cycamore}{\textsc{Cycamore}\xspace}%
\newcommand{\deploy}{\texttt{d3ploy}\xspace}%
\newcommand{\Deploy}{\texttt{D3ploy}\xspace}%
\usepackage{tabularx}
\usepackage{color}
\usepackage{multirow}
\usepackage{float} 
\usepackage[acronym,toc]{glossaries}
%\include{acros}
\definecolor{bg}{rgb}{0.95,0.95,0.95}
\newcolumntype{b}{X}
\newcolumntype{f}{>{\hsize=.15\hsize}X}
\newcolumntype{s}{>{\hsize=.5\hsize}X}
\newcolumntype{m}{>{\hsize=.75\hsize}X}
\newcolumntype{r}{>{\hsize=1.1\hsize}X}
\usepackage{titling}
\usepackage[hang,flushmargin]{footmisc}
\renewcommand*\footnoterule{}
\usepackage{tikz}

\usetikzlibrary{shapes.geometric,arrows}
\tikzstyle{process} = [rectangle, rounded corners, 
minimum width=1cm, minimum height=1cm,text centered, draw=black, 
fill=blue!30]
\tikzstyle{arrow} = [thick,->,>=stealth]

\graphicspath{}
\title{Micro Modular Reactor (MMR) - USNC}
\author{Roberto E. Fairhurst Agosta}

\begin{document}
%	\begin{titlepage}
%		\maketitle
%		\thispagestyle{empty}
%	\end{titlepage}

\section{Description}

The Micro Modular Reactor (MMR) concept integrates power production, power conversion, and electricity generation in a single unit, making it suitable for use in remote areas. 
The MMR uses fully ceramic micro-encapsulated (FCM) fuel, which is a form of TRISO fuel, and He (or CO$_2$) as coolant.
The power of this reactor ranges between 10 to 40 MW-thermal.
The MMR design results attractive for replacing diesel generators in remote areas.
As it could be factory produced, it could generate power at economically competitive prices \cite{hawari_development_2018}.

\section{Serpent2 Model}

\subsection{TRISO particles}

Table \ref{tab:compact} summarizes the characteristics of the TRISO particles and the fuel compact.

The values for the TRISO particles come from Table 5 of \cite{hawari_development_2018}.
This values are not equal to values found in other documents such as in Table 1 of \cite{venneri_neutronic_2015}, but are very close.

While \cite{venneri_neutronic_2015} uses a fuel compact of 0.7 cm of radius, \cite{hawari_development_2018} uses a fuel pin diameter of 2.17 cm.
The model adopts a fuel hole diameter of 2.16 cm.
I assume a helium gap of 0.1 cm between the block and the fuel compact.

Further assumptions:
\begin{itemize}
	\item compact length: 4 cm.
	\item helium temperature: 700 K.
	\item helium pressure: 3 MPa \cite{hawari_development_2018}.
\end{itemize}

Figure \ref{fig:FCM} depicts the fuel compact model in Serpent.

	\begin{figure}[]
		\centering
		\includegraphics[width=0.5\linewidth]{figures/triso1.png}
		\hfill
		\caption{Typical TRISO particle. Image reproduced from \cite{usnc_mmr_2019}.}
		\label{fig:triso}
	\end{figure}

	\begin{table}[]
		\centering
	    \caption{TRISO and Fuel Compact Characteristics.}
	    \label{tab:compact}
		\begin{tabular}{l|l}
		\hline
		Characteristic                   & Value                \\ \hline
		Fuel                             & UO$_2$ + UC$_{0.5}$O$_{1.5}$ (50/50\%)  \\
		Enrichment (average)             & 12.0 wt\%            \\
		Kernel radius                    & 400 $\micro$m        \\
		Buffer thickness                 & 75 $\micro$m         \\
		IPyC thickness                   & 35 $\micro$m         \\
		SiC thickness                    & 35 $\micro$m         \\
		OPyC thickness                   & 20 $\micro$m         \\
    	Kernel density                   & 10.8 g/cm$^3$        \\
		Buffer density                   & 0.98 g/cm$^3$        \\
		IPyC density                     & 1.85 g/cm$^3$        \\
		SiC density                      & 3.2  g/cm$^3$        \\
		OPyC density                     & 1.85 g/cm$^3$        \\
		SiC Matrix density               & 3.2 g/cm$^3$         \\
		Packing Fraction (average)       & 39.9 \%              \\
		Compact diameter                 & 2.14 cm              \\
		Helium gap                       & 0.1 cm               \\
		Compact length                   & 4.0 cm               \\ 
        Helium density           		 & 2.0526 kg/m$^3$      \\
        Block graphite density           & 1.75 g/cm$^3$        \\ \hline

		\end{tabular}
	\end{table}

	\begin{figure}[]
		\centering
		\begin{subfigure}[t]{0.4\textwidth}
			\centering
			\includegraphics[width=\linewidth]{figures/FCM_geom1.png} 
			\caption{XY cross section.}
			\label{fig:FCM_xy}
		\end{subfigure}
		\begin{subfigure}[t]{0.4\textwidth}
			\centering
			\includegraphics[width=\linewidth]{figures/FCM_geom2.png}
			\caption{YZ cross section.}
			\label{fig:FCM_yz}
		\end{subfigure}
		\hfill
		\caption{FCM Serpent model geometry for different planes.}
		\label{fig:FCM}
	\end{figure}



\subsection{Full Core}

\cite{venneri_neutronic_2015} uses a fuel compact of 0.7 cm-radius, a coolant hole of 0.4 cm-radius, and a 17 mm center to center distance hexagonal lattice.
Section 3 of the same document defines the core with graphite pancake-like blocks.
The blocks are 25 cm thick quarter disks with a diameter of 200 cm and a total length of 200 cm, surrounded by a 50 cm thick beryllium oxide reflectors.
The overall core assembly size is 300 cm in diameter and 300 cm in length.

\cite{hawari_development_2018} defines a very different core.
The core definition adopted in this model is closer to \cite{hawari_development_2018} core.
Such core has 37 different hexagonal graphite assemblies.
3 different types assemblies make up the core, the fuel assembly, the control rod assembly, and a central assembly.
The central assembly has one 12 cm-diameter control rod.

%The fuel channel wall and the compacts have a 0.3 cm gap filled with He.

The core has a top and bottom reflector of graphite.
The hexagonal core as a radial reflector of graphite and another one of BeO.

	\begin{table}[]
		\centering
	    \caption{Fuel Assembly Characteristics.}
	    \label{tab:compact}
		\begin{tabular}{l|l}
		\hline
		Characteristic                   & Value         \\ \hline
		Block pitch (flat-to-flat)       & 30 cm         \\
		Block graphite density           & 1.75 g/cm$^3$ \\
		Number of fuel holes             & 54            \\
		Fuel hole diameter               & 2.16 cm       \\
		Compacts per hole                &               \\
		Number of coolant holes          & 19            \\
		Coolant hole radius        		 & 1.54 cm       \\

		Fuel/coolant pitch               &               \\
		Fuel length                      &               \\ \hline

		\end{tabular}
	\end{table}

	\begin{table}[]
		\centering
	    \caption{Control Rod Assembly Characteristics.}
	    \label{tab:compact}
		\begin{tabular}{l|l}
		\hline
		Characteristic                   & Value         \\ \hline
		Block pitch (flat-to-flat)       & 30 cm         \\
		Block graphite density           & 1.75 g/cm$^3$ \\
		Number of fuel holes             & 48            \\
		Fuel hole diameter               & 2.16 cm       \\
		Compacts per hole                &               \\
		Number of coolant holes          & 18            \\
		Large coolant hole radius        & 1.54 cm       \\

		Fuel/coolant pitch               &               \\
		Fuel length                      &               \\ 
		Control rod diameter             & 8 cm          \\ 

		\hline

		\end{tabular}
	\end{table}

\begin{figure}[H]
	\centering
	\includegraphics[width=0.45\linewidth]{figures/MMR_full_stack_geom1.png}
	\hfill
	\caption{XY cross section of the full core Serpent2 model geometry.}
	\label{fig:FullXY}
\end{figure}

\begin{figure}[H]
	\centering
	\includegraphics[width=0.6\linewidth]{figures/MMR_full_stack_geom2.png}
	\hfill
	\caption{YZ cross section of the full core Serpent2 model geometry.}
	\label{fig:FullYZ}
\end{figure}

\begin{figure}[H]
	\centering
	\includegraphics[width=0.4\linewidth]{figures/fuel_block_geom1.png} 
	\hfill
	\caption{Fuel Assembly Serpent2 model geometry.}
	\label{fig:FuelAssembly}
\end{figure}

\begin{figure}[H]
	\centering
	\includegraphics[width=0.4\linewidth]{figures/control_block_geom1.png} 
	\hfill
	\caption{Control Rod Assembly Serpent2 model geometry.}
	\label{fig:ControlRodAssembly}
\end{figure}

\begin{figure}[H]
	\centering
	\includegraphics[width=0.4\linewidth]{figures/central_block_geom1.png}
	\hfill
	\caption{Central Assembly Serpent2 model geometry.}
	\label{fig:CentralAssembly}
\end{figure}

\section{Results}

Both cases use 100 active cycles and 40 inactive cycles. The number of particles is 5000 for each cycle.

The eigenvalue of an elementary cell containing a fuel channel (Fig. \ref{fig:FCM}) is: \\
\noindent
k-eff (analog)    = 1.40963 +/- 0.00164  [1.40641  1.41284]\\
\noindent
k-eff (implicit)  = 1.41108 +/- 0.00090  [1.40932  1.41285]

The eigenvalue of the full core (Fig. \ref{fig:FullXY}) results:\\
\noindent
k-eff (analog)    = 0.79770 +/- 0.00193  [0.79392  0.80148]\\
\noindent
k-eff (implicit)  = 0.79688 +/- 0.00156  [0.79383  0.79993]


\section{Comments}

The results indicate that the model has some erroneous assumption, as the reactor seems to be sub critical.
Many of the values chosen for the model come from \cite{hawari_development_2018}. However, some values are not clearly defined.
For example the dimensions of a TRISO particle and the FCM fuel. Those values came from \cite{powers_fully_2013}, \cite{jo_preliminary_2014}, and \cite{venneri_neutronic_2015}.
Another point that it is worth commenting on is the discrepancy between \cite{hawari_development_2018} and \cite{jo_preliminary_2014} about the core's height. While the first defines it to be around 64 cm, the latter states that a height has to be larger than 200 cm.
The next step will be modify those values to get an eigenvalue greater than 1.

\section{Thermal-Hydraulics}

\begin{figure}[H]
	\centering
	\includegraphics[width=0.5\linewidth]{figures/MMR3D_full.png}
	\hfill
	\caption{3D mesh.}
	\label{fig:gmsh3D}
\end{figure}

\pagebreak
\bibliographystyle{plain}
\bibliography{bibliography}

\end{document}
